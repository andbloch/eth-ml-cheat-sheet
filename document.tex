\documentclass[a4paper,10pt,twoside]{extarticle}

% PARAMETER TO GREY OUT TEXT
\usepackage{etoolbox}
\newtoggle{greytext}
%\toggletrue{greytext}
\togglefalse{greytext}

% PARAMETER TO SHORTEN DOCUMENT
\newtoggle{shortdocument}
\toggletrue{shortdocument}
%\togglefalse{shortdocument}

% REDUCE FONT SIZE IF REQUESTED
\iftoggle{shortdocument}{
% TODO FIND OUT HOW TO SET IT TO 8 pt
}{
}

% DECREASE LINE SPACING IF REQUESTED
\iftoggle{shortdocument}{
\setlength{\baselineskip}{0pt}
\renewcommand{\baselinestretch}{0.00001}
% TODO: next time: (better, but i have to use other equations)
%\setlength{\baselineskip}{1pt}
%\setlength{\lineskiplimit}{-16pt}
}

% PAPER LAYOUT
\iftoggle{shortdocument}{
\usepackage[margin=0.25cm,twoside]{geometry}
}{
\usepackage[margin=0.5cm,twoside,bindingoffset=9mm]{geometry}
}

% LANDSCAPE PACKAGE
\usepackage{pdflscape}

% TODO: page numbering
\usepackage{fancyhdr}
\pagestyle{fancy}
\renewcommand{\headrulewidth}{0pt}
\fancyfoot[C]{\vspace*{-0.55cm}\colorbox{white}{\scriptsize\thepage}}
\setlength{\footskip}{4pt}


% COMMENTS (invisible)
\usepackage{comment}

% COLORS
\usepackage{color}
\usepackage{xcolor}

% COLOR SHORTCUTS
\iftoggle{greytext}{
\newcommand{\tcr}[1]{\textcolor{lighttext}{#1}}
\newcommand{\tcb}[1]{\textcolor{lighttext}{#1}}
\newcommand{\tcg}[1]{\textcolor{lighttext}{#1}}
\newcommand{\tcgr}[1]{\textcolor{lighttext}{#1}}
\newcommand{\tcp}[1]{\textcolor{lighttext}{#1}}
}{
\newcommand{\tcr}[1]{\textcolor{red}{#1}}
\newcommand{\tcb}[1]{\textcolor{blue}{#1}}
\newcommand{\tcg}[1]{\textcolor{green}{#1}}
\newcommand{\tcp}[1]{\textcolor{pink}{#1}}
\newcommand{\tcgr}[1]{\textcolor{gray}{#1}}
}

% MULTICOLUMN LAYOUT
\usepackage{multicol}
\iftoggle{shortdocument}{
\setlength\columnsep{10pt}
}{
\setlength\columnsep{20pt}
}
\setlength{\columnseprule}{0.1pt}
\iftoggle{greytext}{
\renewcommand{\columnseprulecolor}{\color{lighttext}}
}{
}

% PARAGRAPH LAYOUT
\setlength\parindent{0pt}	% indentation
\setlength\parskip{2pt}		% space between paragraphs

% LANGUAGE SETTINGS
%\usepackage[ngerman]{babel} % Silbentrennung (und Deutsche Titel)
\usepackage[utf8]{inputenc} % Umlaute

% FONTS
%\usepackage{mathpazo}
%\usepackage{helvet}
%\usepackage{fouriernc}
%\usepackage[varg]{txfonts}
%\usepackage{mathptmx}
%\usepackage[charter]{mathdesign}
%\usepackage[garamond]{mathdesign}
%\usepackage[utopia]{mathdesign}
%\usepackage{fourier}
% font kerning (load after specific font!)
\usepackage{microtype}

% CUSTOM FONT SIZES
\usepackage{relsize}

% FIGURE SETTINGS
\usepackage{pict2e} % load this before picture
\usepackage{picture}
\usepackage{graphicx}
\usepackage{adjustbox}
\usepackage{caption}
\usepackage{subcaption}

% TIKZ GRAPHICS
\usepackage{tikz}
\usetikzlibrary{calc,matrix,arrows,automata,fit}

% PLOTS
\usepackage{pgfplots}
\pgfplotsset{compat=1.10}

% FLOAT SETTINGS
\usepackage{float}

% REFERENCING
\usepackage{hyperref}

% TABLES
\usepackage{booktabs} % nicer tables
\usepackage{array} % custom column types
\usepackage{multirow} % span cells over multiple rows
\newcommand{\tabitem}{~~\llap{{\boldmath $\cdot$}}~} % items in tables

% LISTS
\usepackage{enumitem}
\setlist{noitemsep,topsep=0pt,parsep=0pt,partopsep=0pt,leftmargin=10pt}
\renewcommand\labelitemi{{\boldmath$\cdot$}}
\renewcommand\labelitemii{{\boldmath$\cdot$}}
\renewcommand\labelitemiii{{\boldmath$\cdot$}}
% advantages disadvantages
\newcommand\pro{\item[$+$]}
\newcommand\con{\item[$-$]}

% TIGHT CENTER ENVIRONMENT
\newenvironment{tightcenter}{%
  \begin{center}
  \vspace{-12pt}
}{
  \vspace{-12pt}
  \end{center}
}

% MATH (packages and settings)
\usepackage{blkarray}  		% matrix annotations
\usepackage[intlimits]{mathtools}
\usepackage{amssymb}
\usepackage{wasysym}
\usepackage{ifsym}
\usepackage{dsfont}
\usepackage{resizegather} 	% resizing equations
\usepackage{oubraces}    	% special overlapping over/underbraces
\usepackage{scalerel}		% scaling of characters
\usepackage{cancel}			% crossing out of terms
\allowdisplaybreaks 		% allow page break in align* environment

% PROOF TREES
\usepackage{proof}
\inferLineSkip=4pt % adjust line skip between proofs

% PROOFS
\newcommand{\qed}{\hfill$\square$}
\newcommand{\contradiction}{\hfill$\lightning$}

% MATH HIGHLIGHTING
\newcommand{\mhl}[2]{\colorbox{#1}{$\displaystyle#2$}}
\newcommand{\mhlr}[1]{\mathhl{red}{#1}}
\newcommand{\mhlg}[1]{\mathhl{green}{#1}}
\newcommand{\mhlb}[1]{\mathhl{blue}{#1}}
\newcommand{\mhly}[1]{\mathhl{yellow}{#1}}

% CODE IN MATH
\newcommand{\mcode}[1]{\texttt{#1}}

% COLOR IN MATH (without the bad behaviour of textcolor)
\makeatletter
\def\mc#1#{\@mc{#1}}
\def\@mc#1#2#3{%
  \protect\leavevmode
  \begingroup
    \color#1{#2}#3%
  \endgroup
}
\makeatother

\iftoggle{greytext}{
\definecolor{brilliantrose}{rgb}{1.0, 0.33, 0.64}
\newcommand{\mcr}[1]{\mc{lighttext}{#1}}
\newcommand{\mcg}[1]{\mc{lighttext}{#1}}
\newcommand{\mcb}[1]{\mc{lighttext}{#1}}
\newcommand{\mcy}[1]{\mc{lighttext}{#1}}
\newcommand{\mcp}[1]{\mc{lighttext}{#1}}
\newcommand{\mcw}[1]{\mc{white}{#1}}
}{
\definecolor{brilliantrose}{rgb}{1.0, 0.33, 0.64}
\newcommand{\mcr}[1]{\mc{red}{#1}}
\newcommand{\mcg}[1]{\mc{green}{#1}}
\newcommand{\mcb}[1]{\mc{blue}{#1}}
\newcommand{\mcy}[1]{\mc{yellow}{#1}}
\newcommand{\mcp}[1]{\mc{brilliantrose}{#1}}
\newcommand{\mcw}[1]{\mc{white}{#1}}
}

% ASYMPTOTIC NOTATIONS
\newcommand{\BigO}{\mathcal{O}}

% NUMBER SYSTEMS
\newcommand{\E}{\mathbb{E}}
\newcommand{\bbH}{\mathbb{H}}
\newcommand{\K}{\mathbb{K}}
\newcommand{\F}{\mathbb{F}}
\newcommand{\N}{\mathbb{N}}
\newcommand{\Z}{\mathbb{Z}}
\newcommand{\Q}{\mathbb{Q}}
\newcommand{\R}{\mathbb{R}}
\newcommand{\M}{\mathbb{M}}
\newcommand{\C}{\mathbb{C}}

% CALLICGRAPHIC SHORTCUTS
\newcommand{\cA}{\mathcal{A}}
\newcommand{\cB}{\mathcal{B}}
\newcommand{\cC}{\mathcal{C}}
\newcommand{\cD}{\mathcal{D}}
\newcommand{\cE}{\mathcal{E}}
\newcommand{\cF}{\mathcal{F}}
\newcommand{\cG}{\mathcal{G}}
\newcommand{\cH}{\mathcal{H}}
\newcommand{\cI}{\mathcal{I}}
\newcommand{\cJ}{\mathcal{J}}
\newcommand{\cK}{\mathcal{K}}
\newcommand{\cL}{\mathcal{L}}
\newcommand{\cM}{\mathcal{M}}
\newcommand{\cN}{\mathcal{N}}
\newcommand{\cO}{\mathcal{O}}
\newcommand{\cP}{\mathcal{P}}
\newcommand{\cQ}{\mathcal{Q}}
\newcommand{\cR}{\mathcal{R}}
\newcommand{\cS}{\mathcal{S}}
\newcommand{\cT}{\mathcal{T}}
\newcommand{\cU}{\mathcal{U}}
\newcommand{\cV}{\mathcal{V}}
\newcommand{\cW}{\mathcal{W}}
\newcommand{\cX}{\mathcal{X}}
\newcommand{\cY}{\mathcal{Y}}
\newcommand{\cZ}{\mathcal{Z}}

% BRACES
\newcommand{\set}[1]{\left\{ #1 \right\}}
\newcommand{\dset}[2]{\left\{ #1 \ \middle| \ #2 \right\}}
\newcommand{\alg}[1]{\left\langle #1 \right\rangle}
\newcommand{\card}[1]{\left\lvert #1 \right\rvert}
\newcommand{\length}[1]{\left\lvert #1 \right\rvert}
\newcommand{\abs}[1]{\left\lvert #1 \right\rvert}
\newcommand{\norm}[1]{\left\lVert #1 \right\rVert}
\newcommand{\scprod}[1]{\left\langle #1 \right\rangle}
\newcommand{\ceil}[1]{\left\lceil #1 \right\rceil}
\newcommand{\floor}[1]{\left\lfloor #1 \right\rfloor}
\newcommand{\linsys}[2]{\left[\ #1 \ \middle| \ #2 \ \right]}
\newcommand{\Sim}[1]{\text{Sim}\left( #1 \right)}

% BRACES SMALL (no adjusting)
\newcommand{\sset}[1]{\{ #1 \}}
\newcommand{\sdset}[2]{\{ #1 \ | \ #2 \}}
\newcommand{\salg}[1]{\langle #1 \rangle}
\newcommand{\scard}[1]{\lvert #1 \rvert}
\newcommand{\slength}[1]{\lvert #1 \rvert}
\newcommand{\sabs}[1]{\lvert #1 \rvert}
\newcommand{\snorm}[1]{\lVert #1 \rVert}
\newcommand{\sscprod}[1]{\langle #1 \rangle}
\newcommand{\sceil}[1]{\lceil #1 \rceil}
\newcommand{\sfloor}[1]{\lfloor #1 \rfloor}
\newcommand{\slinsys}[2]{[\ #1 \ | \ #2 \ ]}
\newcommand{\sSim}[1]{\text{Sim}( #1 )}

% EMPTY SET
\let\oldemptyset\emptyset
\let\emptyset\varnothing

% NOT IN SHORTCUT
\newcommand{\nin}{\not\in}

% DISJOINT UNION SYMBOL
\makeatletter
\def\moverlay{\mathpalette\mov@rlay}
\def\mov@rlay#1#2{\leavevmode\vtop{%
   \baselineskip\z@skip \lineskiplimit-\maxdimen
   \ialign{\hfil$\m@th#1##$\hfil\cr#2\crcr}}}
\newcommand{\charfusion}[3][\mathord]{
    #1{\ifx#1\mathop\vphantom{#2}\fi
        \mathpalette\mov@rlay{#2\cr#3}
      }
    \ifx#1\mathop\expandafter\displaylimits\fi}
\makeatother
\newcommand{\bigcupdot}{\charfusion[\mathop]{\bigcup}{\cdot}}
\newcommand{\cupdot}{\mathbin{\mathaccent\cdot\cup}}

% CUSTOM STATISTICS
\newcommand{\Prob}[2][]{P_{#1}\left( #2 \right)}
\newcommand{\cProb}[2]{P\left( #1 \,\middle|\, #2 \right)}
\newcommand{\Dist}[2]{#1\left( #2 \right)}
\newcommand{\cDist}[3]{#1\left( #2 \,\middle|\, #3 \right)}
\newcommand{\hProb}[2][]{\hat{P}_{#1}\left( #2 \right)}
\newcommand{\chProb}[2]{\hat{P}\left( #1 \,\middle|\, #2 \right)}
\newcommand{\Var}[2][]{\operatorname{Var}_{#1}\left[ #2 \right]}
\newcommand{\sd}[1]{\operatorname{sd}\left( #1 \right)}
\newcommand{\Exp}[2][]{{\mathbb{E}_{#1}}\left[ #2
\right]}
\newcommand{\cExp}[3][]{{\mathbb{E}}_{#1}\left[ #2
\,\middle|\, #3 \right]}
\newcommand{\hExp}[2][]{{\mathbb{\hat{E}}_{#1}}\left[ #2
\right]}
\newcommand{\chExp}[3][]{{\mathbb{\hat{E}}}_{#1}\left[ #2
\,\middle|\, #3 \right]}
\newcommand{\Corr}[1]{\operatorname{Corr}\left[ #1 \right]}
\newcommand{\Cov}[1]{\operatorname{Cov}\left(#1 \right)}
\newcommand{\MSE}[2][]{\operatorname{MSE}_{#1}\left[ #2 \right]}
\newcommand{\riid}{\stackrel{\text{\tiny i.i.d.}}{\sim}}
\newcommand{\approxsim}{\stackrel{\text{approx.}}{\sim}}
\newcommand{\ind}[1]{\mathds{1}_{\set{#1}}}
\newcommand{\eqiid}{\stackrel{\text{\tiny i.i.d.}}{=}}
\newcommand{\eqind}{\stackrel{\text{\tiny ind.}}{=}}

% RANDOM VARIABLES
\newcommand{\rA}{A}
\newcommand{\rB}{B}
\newcommand{\rC}{C}
\newcommand{\rD}{D}
\newcommand{\rE}{E}
\newcommand{\rF}{F}
\newcommand{\rG}{G}
\newcommand{\rH}{H}
\newcommand{\rI}{I}
\newcommand{\rJ}{J}
\newcommand{\rK}{K}
\newcommand{\rL}{L}
\newcommand{\rM}{M}
\newcommand{\rN}{N}
\newcommand{\rO}{O}
\newcommand{\rP}{P}
\newcommand{\rQ}{Q}
\newcommand{\rR}{R}
\newcommand{\rS}{S}
\newcommand{\rT}{T}
\newcommand{\rU}{U}
\newcommand{\rV}{V}
\newcommand{\rW}{W}
\newcommand{\rX}{X}
\newcommand{\rY}{Y}
\newcommand{\rZ}{Z}

% RANDOM VECTORS
% declares a custom italic bold alphabet for random vectors
\DeclareMathAlphabet{\mathbfit}{OML}{cmm}{b}{it}
\newcommand{\rvA}{\mathbfit{A}}
\newcommand{\rvB}{\mathbfit{B}}
\newcommand{\rvC}{\mathbfit{C}}
\newcommand{\rvD}{\mathbfit{D}}
\newcommand{\rvE}{\mathbfit{E}}
\newcommand{\rvF}{\mathbfit{F}}
\newcommand{\rvG}{\mathbfit{G}}
\newcommand{\rvH}{\mathbfit{H}}
\newcommand{\rvI}{\mathbfit{I}}
\newcommand{\rvJ}{\mathbfit{J}}
\newcommand{\rvK}{\mathbfit{K}}
\newcommand{\rvL}{\mathbfit{L}}
\newcommand{\rvM}{\mathbfit{M}}
\newcommand{\rvN}{\mathbfit{N}}
\newcommand{\rvO}{\mathbfit{O}}
\newcommand{\rvP}{\mathbfit{P}}
\newcommand{\rvQ}{\mathbfit{Q}}
\newcommand{\rvR}{\mathbfit{R}}
\newcommand{\rvS}{\mathbfit{S}}
\newcommand{\rvT}{\mathbfit{T}}
\newcommand{\rvU}{\mathbfit{U}}
\newcommand{\rvV}{\mathbfit{V}}
\newcommand{\rvW}{\mathbfit{W}}
\newcommand{\rvX}{\mathbfit{X}}
\newcommand{\rvY}{\mathbfit{Y}}
\newcommand{\rvZ}{\mathbfit{Z}}

% MACHINE LEARNING
\newcommand{\Risk}[1]{R\left(#1\right)}
\newcommand{\empRisk}[1]{\widehat{R}\left(#1\right)}

% ACCENTS
% TODO: fix this, make spacing nice
\newcommand*{\Hm}{\mathsf{H}}
\newcommand*{\T}{\mathsf{T}}
%\newcommand*{\T}{\mkern-1mu{}_{}^{\scriptscriptstyle\top}\mkern-4mu}
\newcommand*{\Rev}{\mathsf{R}}
\newcommand{\conj}[1]{\overline{ #1 }}

% CUSTOM ALPHABETS
\renewcommand{\S}{\Sigma}
\newcommand{\Ss}{\Sigma^*}
\newcommand{\Sp}{\Sigma^+}
\newcommand{\Sbool}{\Sigma_{\text{bool}}}
\newcommand{\Ssbool}{(\Sigma_{\text{bool}})^*}
\newcommand{\Slogic}{\Sigma_{\text{logic}}}
\newcommand{\Sslogic}{(\Sigma_{\text{logic}})^*}
\newcommand{\Slat}{\Sigma_{\text{lat}}}
\newcommand{\Sslat}{(\Sigma_{\text{lat}})^*}
\newcommand{\Stastatur}{\Sigma_{\text{Tastatur}}}
\newcommand{\Sstastatur}{(\Sigma_{\text{Tastatur}})^*}
\newcommand{\Sm}{\Sigma_{m}}
\newcommand{\Ssm}{\Sigma_{m}^*}
\newcommand{\ZO}{\{0,1\}}
\newcommand{\ZOs}{\{0,1\}^*}
\newcommand{\hdelta}{\hat\delta}

% OPERATORS
% TODO: Should I design these as braces?
\DeclareMathOperator{\id}{\text{id}}
\DeclareMathOperator{\Kon}{\text{Kon}}
\DeclareMathOperator{\cost}{\text{cost}}
\DeclareMathOperator{\goal}{\text{goal}}
\DeclareMathOperator{\Opt}{\text{Opt}}
\DeclareMathOperator{\Bin}{\text{Bin}}
\DeclareMathOperator{\Nummer}{\text{Nummer}}
\DeclareMathOperator{\Prim}{\text{Prim}}
\DeclareMathOperator{\Kl}{\text{Kl}}
\DeclareMathOperator{\lcm}{lcm}
\DeclareMathOperator{\glb}{glb}
\DeclareMathOperator{\lub}{lub}
\DeclareMathOperator{\im}{im}
\DeclareMathOperator{\ord}{ord}
\DeclareMathOperator{\rank}{rank}
\DeclareMathOperator{\spn}{span}
\DeclareMathOperator{\Cost}{Cost}
\DeclareMathOperator{\order}{order}
\DeclareMathOperator{\dist}{dist}
\DeclareMathOperator{\cond}{cond}
\DeclareMathOperator{\nnz}{nnz}
\DeclareMathOperator{\sign}{sign}
\DeclareMathOperator{\Tr}{Tr}
\DeclareMathOperator{\Count}{Count}
\DeclareMathOperator{\Spur}{Spur}
\DeclareMathOperator{\diag}{diag}
\DeclareMathOperator{\triu}{triu}
\DeclareMathOperator{\cumsum}{cumsum}
\DeclareMathOperator{\vectorize}{vectorize}
\DeclareMathOperator{\matrixfy}{matrixfy}
\DeclareMathOperator{\circul}{circul}
\DeclareMathOperator{\dft}{dft}
\DeclareMathOperator{\invdft}{invdft}
\DeclareMathOperator{\ones}{ones}
\DeclareMathOperator{\arcsinh}{arcsinh}
\DeclareMathOperator{\arccosh}{arccosh}
\DeclareMathOperator{\arctanh}{arctanh}
\renewcommand\div{\operatorname{div}}
\DeclareMathOperator{\rot}{rot}
\DeclareMathOperator{\cis}{cis}
\DeclareMathOperator{\grad}{grad}
\DeclareMathOperator{\Hess}{Hess}
\newcommand{\laplace}{\Delta}
\DeclareMathOperator*{\argmin}{arg\,min}
\DeclareMathOperator*{\argmax}{arg\,max}
\DeclareMathOperator{\odd}{odd}
\DeclareMathOperator{\even}{even}
\DeclareMathOperator{\Proj}{Proj}

% CRYPTOGRAPHY
\DeclareMathOperator{\concat}{ || }
\DeclareMathOperator{\Enc}{Enc}
\DeclareMathOperator{\Dec}{Dec}
\DeclareMathOperator{\Gen}{Gen}
\DeclareMathOperator{\Tag}{Tag}
\DeclareMathOperator{\Vrfy}{Vrfy}
\DeclareMathOperator{\MAC}{\text{MAC}}
\newcommand{\AdvPRG}[2][]{\text{Adv}_{\text{PRG}}\left[ #2 \right]}
\newcommand{\yes}{\text{yes}}
\newcommand{\no}{\text{no}}
\newcommand{\forallPPTTM}{%
\underset{\mathclap{\substack{\text{\tiny prob. poly-}\\
\text{\tiny time TM}}}}{\forall}}
\newcommand{\forallPTAdv}{%
\underset{\mathclap{\substack{\text{\tiny poly-time}\\
\text{\tiny Adversaries}}}}{\forall}}

% OPERATORS (OVERRIDDEN)
\renewcommand\Re{\operatorname{Re}}
\renewcommand\Im{\operatorname{Im}}

% RELATIONAL ALGEBRA 
\DeclareMathOperator{\dom}{dom}
\DeclareMathOperator{\sch}{sch}
\newcommand{\dto}{\stackrel{\bullet}{\to}}
\newcommand{\join}{\bowtie}
\newcommand{\lojoin}{{\tiny \textifsym{d|><|}}}
\newcommand{\rojoin}{{\tiny \textifsym{|><|d}}}
\newcommand{\fojoin}{{\tiny \textifsym{d|><|d}}}
\newcommand{\lsjoin}{\ltimes}
\newcommand{\rsjoin}{\rtimes}

% RELATIONS
\newcommand{\mbeq}{\stackrel{!}{=}}
\newcommand{\xor}{\mathrel{\text{xor}}}
\newcommand{\relid}{\mathrel{\id}}
\newcommand{\relrho}{\mathrel{\rho}}
\newcommand{\relsigma}{\mathrel{\sigma}}
\newcommand{\reltheta}{\mathrel{\theta}}
\newcommand{\relsim}{\mathrel{\sim}}
\newcommand{\relf}{\mathrel{f}}

% RELATIONS (INVERSES)
\newcommand{\invrelid}{\mathrel{\widehat{\id}}}
\newcommand{\invrelrho}{\mathrel{\widehat{\rho}}}
\newcommand{\invrelsigma}{\mathrel{\widehat{\sigma}}}
\newcommand{\invreltheta}{\mathrel{\widehat{\theta}}}
\newcommand{\invrelsim}{\mathrel{\widehat{\sim}}}
\newcommand{\invrelf}{\mathrel{\widehat{f}}}

% CUSTOM RELATIONS
\DeclareRobustCommand{\step}[2][]{\mathrel{\drawstep{#1}{#2}}}
\newcommand{\drawstep}[2]{%
  \vcenter{\hbox{%
    \setlength{\unitlength}{1em}%
    \begin{picture}(1,1)
    \roundcap
    \put(0,0){\line(0,1){1}}
    \put(0,0.5){\line(1,0){0.95}}
    \put(0.5,0){\makebox[0pt]{\text{\smaller$\scriptscriptstyle#2$}}}
    \put(0.5,0.6){\makebox[0pt]{\text{\smaller$#1$}}}
    \end{picture}%
  }}%
}

% LINEAR TEMPORAL LOGIC (LTL)
\newcommand{\until}{\texttt{\,\hstretch{0.7}{\boldsymbol{\cup}}\,}}
\newcommand{\next}{\Circle}
\newcommand{\eventually}{\Diamond}
\newcommand{\always}{\square}

% GLOBAL MATRICES AND VECTOR SETTINGS
\newcommand{\boldm}[1] {\mathversion{bold}#1\mathversion{normal}}
\newcommand{\mat}[1]{\mathbf{#1}}
\renewcommand{\vec}[1]{\mathbf{#1}}

% VECTORS (LATIN)
\newcommand{\va}{\vec{a}}
\newcommand{\vb}{\vec{b}}
\newcommand{\vc}{\vec{c}}
\newcommand{\vd}{\vec{d}}
\newcommand{\ve}{\vec{e}}
\newcommand{\vf}{\vec{f}}
\newcommand{\vg}{\vec{g}}
\newcommand{\vh}{\vec{h}}
\newcommand{\vi}{\vec{i}}
\newcommand{\vj}{\vec{j}}
\newcommand{\vk}{\vec{k}}
\newcommand{\vl}{\vec{l}}
\newcommand{\vm}{\vec{m}}
\newcommand{\vn}{\vec{n}}
\newcommand{\vo}{\vec{o}}
\newcommand{\vp}{\vec{p}}
\newcommand{\vq}{\vec{q}}
\newcommand{\vr}{\vec{r}}
\newcommand{\vs}{\vec{s}}
\newcommand{\vt}{\vec{t}}
\newcommand{\vu}{\vec{u}}
\newcommand{\vv}{\vec{v}}
\newcommand{\vw}{\vec{w}}
\newcommand{\vx}{\vec{x}}
\newcommand{\vy}{\vec{y}}
\newcommand{\vz}{\vec{z}}

% VECTORS (LATIN) WITH TILDE ACCENT
\newcommand{\vta}{\widetilde{\vec{a}}}
\newcommand{\vtb}{\widetilde{\vec{b}}}
\newcommand{\vtc}{\widetilde{\vec{c}}}
\newcommand{\vtd}{\widetilde{\vec{d}}}
\newcommand{\vte}{\widetilde{\vec{e}}}
\newcommand{\vtf}{\widetilde{\vec{f}}}
\newcommand{\vtg}{\widetilde{\vec{g}}}
\newcommand{\vth}{\widetilde{\vec{h}}}
\newcommand{\vti}{\widetilde{\vec{i}}}
\newcommand{\vtj}{\widetilde{\vec{j}}}
\newcommand{\vtk}{\widetilde{\vec{k}}}
\newcommand{\vtl}{\widetilde{\vec{l}}}
\newcommand{\vtm}{\widetilde{\vec{m}}}
\newcommand{\vtn}{\widetilde{\vec{n}}}
\newcommand{\vto}{\widetilde{\vec{o}}}
\newcommand{\vtp}{\widetilde{\vec{p}}}
\newcommand{\vtq}{\widetilde{\vec{q}}}
\newcommand{\vtr}{\widetilde{\vec{r}}}
\newcommand{\vts}{\widetilde{\vec{s}}}
\newcommand{\vtt}{\widetilde{\vec{t}}}
\newcommand{\vtu}{\widetilde{\vec{u}}}
\newcommand{\vtv}{\widetilde{\vec{v}}}
\newcommand{\vtw}{\widetilde{\vec{w}}}
\newcommand{\vtx}{\widetilde{\vec{x}}}
\newcommand{\vty}{\widetilde{\vec{y}}}
\newcommand{\vtz}{\widetilde{\vec{z}}}

% VECTORS (LATIN) WITH HAT ACCENT
\newcommand{\vha}{\widehat{\vec{a}}}
\newcommand{\vhb}{\widehat{\vec{b}}}
\newcommand{\vhc}{\widehat{\vec{c}}}
\newcommand{\vhd}{\widehat{\vec{d}}}
\newcommand{\vhe}{\widehat{\vec{e}}}
\newcommand{\vhf}{\widehat{\vec{f}}}
\newcommand{\vhg}{\widehat{\vec{g}}}
\newcommand{\vhh}{\widehat{\vec{h}}}
\newcommand{\vhi}{\widehat{\vec{i}}}
\newcommand{\vhj}{\widehat{\vec{j}}}
\newcommand{\vhk}{\widehat{\vec{k}}}
\newcommand{\vhl}{\widehat{\vec{l}}}
\newcommand{\vhm}{\widehat{\vec{m}}}
\newcommand{\vhn}{\widehat{\vec{n}}}
\newcommand{\vho}{\widehat{\vec{o}}}
\newcommand{\vhp}{\widehat{\vec{p}}}
\newcommand{\vhq}{\widehat{\vec{q}}}
\newcommand{\vhr}{\widehat{\vec{r}}}
\newcommand{\vhs}{\widehat{\vec{s}}}
\newcommand{\vht}{\widehat{\vec{t}}}
\newcommand{\vhu}{\widehat{\vec{u}}}
\newcommand{\vhv}{\widehat{\vec{v}}}
\newcommand{\vhw}{\widehat{\vec{w}}}
\newcommand{\vhx}{\widehat{\vec{x}}}
\newcommand{\vhy}{\widehat{\vec{y}}}
\newcommand{\vhz}{\widehat{\vec{z}}}

% VECTORS (GREEK)
\newcommand{\valpha}{\boldsymbol{\alpha}}
\newcommand{\vbeta}{\boldsymbol{\beta}}
\newcommand{\vgamma}{\boldsymbol{\gamma}}
\newcommand{\vdelta}{\boldsymbol{\delta}}
\newcommand{\vepsilon}{\boldsymbol{\epsilon}}
\newcommand{\vvarepsilon}{\boldsymbol{\varepsilon}}
\newcommand{\vzeta}{\boldsymbol{\zeta}}
\newcommand{\veta}{\boldsymbol{\eta}}
\newcommand{\vtheta}{\boldsymbol{\theta}}
\newcommand{\viota}{\boldsymbol{\iota}}
\newcommand{\vkappa}{\boldsymbol{\kappa}}
\newcommand{\vlambda}{\boldsymbol{\lambda}}
\newcommand{\vmu}{\boldsymbol{\mu}}
\newcommand{\vnu}{\boldsymbol{\nu}}
\newcommand{\vxi}{\boldsymbol{\xi}}
% omikron is just latin 'o'
\newcommand{\vpi}{\boldsymbol{\pi}}
\newcommand{\vrho}{\boldsymbol{\rho}}
\newcommand{\vsigma}{\boldsymbol{\sigma}}
\newcommand{\vtau}{\boldsymbol{\tau}}
\newcommand{\vupsilon}{\boldsymbol{\upsilon}}
\newcommand{\vphi}{\boldsymbol{\phi}}
\newcommand{\vvarphi}{\boldsymbol{\varphi}}
\newcommand{\vchi}{\boldsymbol{\chi}}
\newcommand{\vpsi}{\boldsymbol{\psi}}
\newcommand{\vomega}{\boldsymbol{\omega}}

% VECTORS (GREEK) WITH TILDE ACCENT
\newcommand{\vtalpha}{\widetilde{\valpha}}
\newcommand{\vtbeta}{\widetilde{\vbeta}}
\newcommand{\vtgamma}{\widetilde{\vgamma}}
\newcommand{\vtdelta}{\widetilde{\vdelta}}
\newcommand{\vtepsilon}{\widetilde{\vepsilon}}
\newcommand{\vtvarepsilon}{\widetilde{\vvarepsilon}}
\newcommand{\vtzeta}{\widetilde{\vzeta}}
\newcommand{\vteta}{\widetilde{\veta}}
\newcommand{\vttheta}{\widetilde{\vtheta}}
\newcommand{\vtiota}{\widetilde{\viota}}
\newcommand{\vtkappa}{\widetilde{\vkappa}}
\newcommand{\vtlambda}{\widetilde{\vlambda}}
\newcommand{\vtmu}{\widetilde{\vmu}}
\newcommand{\vtnu}{\widetilde{\vnu}}
\newcommand{\vtxi}{\widetilde{\vxi}}
% omikron is just latin 'o'
\newcommand{\vtpi}{\widetilde{\vpi}}
\newcommand{\vtrho}{\widetilde{\vrho}}
\newcommand{\vtsigma}{\widetilde{\vsigma}}
\newcommand{\vttau}{\widetilde{\vtau}}
\newcommand{\vtupsilon}{\widetilde{\vupsilon}}
\newcommand{\vtphi}{\widetilde{\vphi}}
\newcommand{\vtvarphi}{\widetilde{\vvarphi}}
\newcommand{\vtchi}{\widetilde{\vchi}}
\newcommand{\vtpsi}{\widetilde{\vpsi}}
\newcommand{\vtomega}{\widetilde{\vomega}}

% VECTORS (GREEK) WITH HAT ACCENT
\newcommand{\vhalpha}{\widehat{\valpha}}
\newcommand{\vhbeta}{\widehat{\vbeta}}
\newcommand{\vhgamma}{\widehat{\vgamma}}
\newcommand{\vhdelta}{\widehat{\vdelta}}
\newcommand{\vhepsilon}{\widehat{\vepsilon}}
\newcommand{\vhvarepsilon}{\widehat{\vvarepsilon}}
\newcommand{\vhzeta}{\widehat{\vzeta}}
\newcommand{\vheta}{\widehat{\veta}}
\newcommand{\vhtheta}{\widehat{\vtheta}}
\newcommand{\vhiota}{\widehat{\viota}}
\newcommand{\vhkappa}{\widehat{\vkappa}}
\newcommand{\vhlambda}{\widehat{\vlambda}}
\newcommand{\vhmu}{\widehat{\vmu}}
\newcommand{\vhnu}{\widehat{\vnu}}
\newcommand{\vhxi}{\widehat{\vxi}}
% omikron is just latin 'o'
\newcommand{\vhpi}{\widehat{\vpi}}
\newcommand{\vhrho}{\widehat{\vrho}}
\newcommand{\vhsigma}{\widehat{\vsigma}}
\newcommand{\vhtau}{\widehat{\vthau}}
\newcommand{\vhupsilon}{\widehat{\vupsilon}}
\newcommand{\vhphi}{\widehat{\vphi}}
\newcommand{\vhvarphi}{\widehat{\vvarphi}}
\newcommand{\vhchi}{\widehat{\vchi}}
\newcommand{\vhpsi}{\widehat{\vpsi}}
\newcommand{\vhomega}{\widehat{\vomega}}

% MATRICES (LATIN)
\newcommand{\MA}{\mat{A}}
\newcommand{\MB}{\mat{B}}
\newcommand{\MC}{\mat{C}}
\newcommand{\MD}{\mat{D}}
\newcommand{\ME}{\mat{E}}
\newcommand{\MF}{\mat{F}}
\newcommand{\MG}{\mat{G}}
\newcommand{\MH}{\mat{H}}
\newcommand{\MI}{\mat{I}}
\newcommand{\MJ}{\mat{J}}
\newcommand{\MK}{\mat{K}}
\newcommand{\ML}{\mat{L}}
\newcommand{\MM}{\mat{M}}
\newcommand{\MN}{\mat{N}}
\newcommand{\MO}{\mat{0}}
\newcommand{\MP}{\mat{P}}
\newcommand{\MQ}{\mat{Q}}
\newcommand{\MR}{\mat{R}}
\newcommand{\MS}{\mat{S}}
\newcommand{\MT}{\mat{T}}
\newcommand{\MU}{\mat{U}}
\newcommand{\MV}{\mat{V}}
\newcommand{\MW}{\mat{W}}
\newcommand{\MX}{\mat{X}}
\newcommand{\MY}{\mat{Y}}
\newcommand{\MZ}{\mat{Z}}

% MATRICES (LATIN) TILDE
\newcommand{\MtA}{\widetilde{\mat{A}}}
\newcommand{\MtB}{\widetilde{\mat{B}}}
\newcommand{\MtC}{\widetilde{\mat{C}}}
\newcommand{\MtD}{\widetilde{\mat{D}}}
\newcommand{\MtE}{\widetilde{\mat{E}}}
\newcommand{\MtF}{\widetilde{\mat{F}}}
\newcommand{\MtG}{\widetilde{\mat{G}}}
\newcommand{\MtH}{\widetilde{\mat{H}}}
\newcommand{\MtI}{\widetilde{\mat{I}}}
\newcommand{\MtJ}{\widetilde{\mat{J}}}
\newcommand{\MtK}{\widetilde{\mat{K}}}
\newcommand{\MtL}{\widetilde{\mat{L}}}
\newcommand{\MtM}{\widetilde{\mat{M}}}
\newcommand{\MtN}{\widetilde{\mat{N}}}
\newcommand{\MtO}{\widetilde{\mat{0}}}
\newcommand{\MtP}{\widetilde{\mat{P}}}
\newcommand{\MtQ}{\widetilde{\mat{Q}}}
\newcommand{\MtR}{\widetilde{\mat{R}}}
\newcommand{\MtS}{\widetilde{\mat{S}}}
\newcommand{\MtT}{\widetilde{\mat{T}}}
\newcommand{\MtU}{\widetilde{\mat{U}}}
\newcommand{\MtV}{\widetilde{\mat{V}}}
\newcommand{\MtW}{\widetilde{\mat{W}}}
\newcommand{\MtX}{\widetilde{\mat{X}}}
\newcommand{\MtY}{\widetilde{\mat{Y}}}
\newcommand{\MtZ}{\widetilde{\mat{Z}}}

% MATRICES (LATIN) HAT
\newcommand{\MhA}{\widehat{\mat{A}}}
\newcommand{\MhB}{\widehat{\mat{B}}}
\newcommand{\MhC}{\widehat{\mat{C}}}
\newcommand{\MhD}{\widehat{\mat{D}}}
\newcommand{\MhE}{\widehat{\mat{E}}}
\newcommand{\MhF}{\widehat{\mat{F}}}
\newcommand{\MhG}{\widehat{\mat{G}}}
\newcommand{\MhH}{\widehat{\mat{H}}}
\newcommand{\MhI}{\widehat{\mat{I}}}
\newcommand{\MhJ}{\widehat{\mat{J}}}
\newcommand{\MhK}{\widehat{\mat{K}}}
\newcommand{\MhL}{\widehat{\mat{L}}}
\newcommand{\MhM}{\widehat{\mat{M}}}
\newcommand{\MhN}{\widehat{\mat{N}}}
\newcommand{\MhO}{\widehat{\mat{0}}}
\newcommand{\MhP}{\widehat{\mat{P}}}
\newcommand{\MhQ}{\widehat{\mat{Q}}}
\newcommand{\MhR}{\widehat{\mat{R}}}
\newcommand{\MhS}{\widehat{\mat{S}}}
\newcommand{\MhT}{\widehat{\mat{T}}}
\newcommand{\MhU}{\widehat{\mat{U}}}
\newcommand{\MhV}{\widehat{\mat{V}}}
\newcommand{\MhW}{\widehat{\mat{W}}}
\newcommand{\MhX}{\widehat{\mat{X}}}
\newcommand{\MhY}{\widehat{\mat{Y}}}
\newcommand{\MhZ}{\widehat{\mat{Z}}}

% MATRICES (GREEK)
\newcommand{\MGamma}{\mat{\Gamma}}
\newcommand{\MDelta}{\mat{\Delta}}
\newcommand{\MTheta}{\mat{\Theta}}
\newcommand{\MLambda}{\mat{\Lambda}}
\newcommand{\MXi}{\mat{\Xi}}
\newcommand{\MPi}{\mat{\Pi}}
\newcommand{\MSigma}{\mat{\Sigma}}
\newcommand{\MUpsilon}{\mat{\Upsilon}}
\newcommand{\MPhi}{\mat{\Phi}}
\newcommand{\MPsi}{\mat{\Psi}}
\newcommand{\MOmega}{\mat{\Omega}}

% MATRICES (GREEK) TILDE
\newcommand{\MtGamma}{\widetilde{\MGamma}}
\newcommand{\MtDelta}{\widetilde{\MDelta}}
\newcommand{\MtTheta}{\widetilde{\MTheta}}
\newcommand{\MtLambda}{\widetilde{\MLambda}}
\newcommand{\MtXi}{\widetilde{\MXi}}
\newcommand{\MtPi}{\widetilde{\MPi}}
\newcommand{\MtSigma}{\widetilde{\MSigma}}
\newcommand{\MtUpsilon}{\widetilde{\MUpsilon}}
\newcommand{\MtPhi}{\widetilde{\MPhi}}
\newcommand{\MtPsi}{\widetilde{\MPsi}}
\newcommand{\MtOmega}{\widetilde{\MOmega}}

% MATRICES (GREEK) HAT
\newcommand{\MhGamma}{\widehat{\MGamma}}
\newcommand{\MhDelta}{\widehat{\MDelta}}
\newcommand{\MhTheta}{\widehat{\MTheta}}
\newcommand{\MhLambda}{\widehat{\MLambda}}
\newcommand{\MhXi}{\widehat{\MXi}}
\newcommand{\MhPi}{\widehat{\MPi}}
\newcommand{\MhSigma}{\widehat{\MSigma}}
\newcommand{\MhUpsilon}{\widehat{\MUpsilon}}
\newcommand{\MhPhi}{\widehat{\MPhi}}
\newcommand{\MhPsi}{\widehat{\MPsi}}
\newcommand{\MhOmega}{\widehat{\MOmega}}

% MATRIX SPACING BARS (for representing row/column-vectors)
\newcommand*{\vertbar}{\rule[-1ex]{0.5pt}{4ex}}
\newcommand*{\horzbar}{\rule[.5ex]{4ex}{0.5pt}}
\newcommand*{\svertbar}{\rule[-0.5ex]{0.5pt}{2ex}}

% CUSTOM NUMERICS
\newcommand{\EPS}{\text{EPS}}
\DeclareMathOperator{\rd}{rd}
\newcommand{\op}{\mathbin{\text{op}}}
\newcommand{\mop}{\mathbin{\widetilde{\text{op}}}}


% TILDE CHARACTERS
\newcommand{\wta}{\widetilde{a}}
\newcommand{\wtb}{\widetilde{b}}
\newcommand{\wtc}{\widetilde{c}}
\newcommand{\wtd}{\widetilde{d}}
\newcommand{\wte}{\widetilde{e}}
\newcommand{\wtf}{\widetilde{f}}
\newcommand{\wtg}{\widetilde{g}}
\newcommand{\wth}{\widetilde{h}}
\newcommand{\wti}{\widetilde{i}}
\newcommand{\wtj}{\widetilde{j}}
\newcommand{\wtk}{\widetilde{k}}
\newcommand{\wtl}{\widetilde{l}}
\newcommand{\wtm}{\widetilde{m}}
\newcommand{\wtn}{\widetilde{n}}
\newcommand{\wto}{\widetilde{o}}
\newcommand{\wtp}{\widetilde{p}}
\newcommand{\wtq}{\widetilde{q}}
\newcommand{\wtr}{\widetilde{r}}
\newcommand{\wts}{\widetilde{s}}
\newcommand{\wtt}{\widetilde{t}}
\newcommand{\wtu}{\widetilde{u}}
\newcommand{\wtv}{\widetilde{v}}
\newcommand{\wtw}{\widetilde{w}}
\newcommand{\wtx}{\widetilde{x}}
\newcommand{\wty}{\widetilde{y}}
\newcommand{\wtz}{\widetilde{z}}
\newcommand{\wtA}{\widetilde{A}}
\newcommand{\wtB}{\widetilde{B}}
\newcommand{\wtC}{\widetilde{C}}
\newcommand{\wtD}{\widetilde{D}}
\newcommand{\wtE}{\widetilde{E}}
\newcommand{\wtF}{\widetilde{F}}
\newcommand{\wtG}{\widetilde{G}}
\newcommand{\wtH}{\widetilde{H}}
\newcommand{\wtI}{\widetilde{I}}
\newcommand{\wtJ}{\widetilde{J}}
\newcommand{\wtK}{\widetilde{K}}
\newcommand{\wtL}{\widetilde{L}}
\newcommand{\wtM}{\widetilde{M}}
\newcommand{\wtN}{\widetilde{N}}
\newcommand{\wtO}{\widetilde{O}}
\newcommand{\wtP}{\widetilde{P}}
\newcommand{\wtQ}{\widetilde{Q}}
\newcommand{\wtR}{\widetilde{R}}
\newcommand{\wtS}{\widetilde{S}}
\newcommand{\wtT}{\widetilde{T}}
\newcommand{\wtU}{\widetilde{U}}
\newcommand{\wtV}{\widetilde{V}}
\newcommand{\wtW}{\widetilde{W}}
\newcommand{\wtX}{\widetilde{X}}
\newcommand{\wtY}{\widetilde{Y}}
\newcommand{\wtZ}{\widetilde{Z}}

% HAT CHARACTERS
\newcommand{\wha}{\widehat{a}}
\newcommand{\whb}{\widehat{b}}
\newcommand{\whc}{\widehat{c}}
\newcommand{\whd}{\widehat{d}}
\newcommand{\whe}{\widehat{e}}
\newcommand{\whf}{\widehat{f}}
\newcommand{\whg}{\widehat{g}}
\newcommand{\whh}{\widehat{h}}
\newcommand{\whi}{\widehat{i}}
\newcommand{\whj}{\widehat{j}}
\newcommand{\whk}{\widehat{k}}
\newcommand{\whl}{\widehat{l}}
\newcommand{\whm}{\widehat{m}}
\newcommand{\whn}{\widehat{n}}
\newcommand{\who}{\widehat{o}}
\newcommand{\whp}{\widehat{p}}
\newcommand{\whq}{\widehat{q}}
\newcommand{\whr}{\widehat{r}}
\newcommand{\whs}{\widehat{s}}
\newcommand{\wht}{\widehat{t}}
\newcommand{\whu}{\widehat{u}}
\newcommand{\whv}{\widehat{v}}
\newcommand{\whw}{\widehat{w}}
\newcommand{\whx}{\widehat{x}}
\newcommand{\why}{\widehat{y}}
\newcommand{\whz}{\widehat{z}}
\newcommand{\whA}{\widehat{A}}
\newcommand{\whB}{\widehat{B}}
\newcommand{\whC}{\widehat{C}}
\newcommand{\whD}{\widehat{D}}
\newcommand{\whE}{\widehat{E}}
\newcommand{\whF}{\widehat{F}}
\newcommand{\whG}{\widehat{G}}
\newcommand{\whH}{\widehat{H}}
\newcommand{\whI}{\widehat{I}}
\newcommand{\whJ}{\widehat{J}}
\newcommand{\whK}{\widehat{K}}
\newcommand{\whL}{\widehat{L}}
\newcommand{\whM}{\widehat{M}}
\newcommand{\whN}{\widehat{N}}
\newcommand{\whO}{\widehat{O}}
\newcommand{\whP}{\widehat{P}}
\newcommand{\whQ}{\widehat{Q}}
\newcommand{\whR}{\widehat{R}}
\newcommand{\whS}{\widehat{S}}
\newcommand{\whT}{\widehat{T}}
\newcommand{\whU}{\widehat{U}}
\newcommand{\whV}{\widehat{V}}
\newcommand{\whW}{\widehat{W}}
\newcommand{\whX}{\widehat{X}}
\newcommand{\whY}{\widehat{Y}}
\newcommand{\whZ}{\widehat{Z}}

% ARGUMENT DOT
\newcommand{\argdot}{\,\cdot\,}

% BOXED TEXT
\usepackage{environ}
\NewEnviron{textbox}{
\begin{center}
\fbox{\parbox{0.9\linewidth}{%
\BODY
}}
\end{center}
}

% ALGORITHMS
\usepackage[ruled,vlined]{algorithm2e}
\SetKwProg{Fn}{Function}{}{}
\SetKwComment{Comment}{$\triangleright$\ }{}
\SetKwInput{KwIn}{In}
\SetKwInput{KwOut}{Out}
\SetKw{KwBreak}{break}

% PROGRAM CODES
% TODO DEFINE Programming language specific environments
% which already include all the options (mathescape, etc.)
\usepackage{listings}
\usepackage{courier}

% global program code settings
\lstset{
 language          = C++, % TODO: see above
 basicstyle        = \bfseries\footnotesize\ttfamily,
 keywordstyle      = \color{blue}\footnotesize\bfseries\ttfamily,
 stringstyle       = \color{red}\footnotesize\ttfamily,
 commentstyle      = \color{magenta}\footnotesize\ttfamily,
 morecomment       = [l][\color{magenta}]{\#},
 showstringspaces  = false,
 breaklines        = true,
 breakatwhitespace = true,
 breakindent       = 2ex,
 tabsize           = 2,
 mathescape        = true
}
% global inline code settings
\newcommand\inlinestyle{\lstset{
 basicstyle        = \bfseries\small\ttfamily,
 keywordstyle      = \color{blue}\bfseries\small\ttfamily,
 stringstyle       = \color{red}\small\ttfamily,
 commentstyle      = \color{magenta}\small\ttfamily
}}

% pseudo code environment
% TODO: add math escaping to true here
\lstnewenvironment{Code}[1][]{}{}        % boxed
\newcommand{\code}[1]{{\inlinestyle\lstinline{#1}}}    % inline

% C/C++ style for highlighting
\newcommand\cppstyle{\lstset{
 language=C++
}}
% C/C++ environments
\lstnewenvironment{Cpp}[1][]{\cppstyle \lstset{#1}}{}
\newcommand\Cppexternal[2][]{{\cppstyle \lstinputlisting[#1]{#2}}}
\newcommand\cpp[1]{{\cppstyle\inlinestyle\lstinline!#1!}}

% Python style for highlighting
\newcommand\pythonstyle{\lstset{
 language=Python,
 otherkeywords={self}             % Add keywords here
}}
% Python environments
\lstnewenvironment{Python}[1][]{\pythonstyle \lstset{#1}}{}
\newcommand\Pythonexternal[2][]{{\pythonstyle \lstinputlisting[#1]{#2}}}
\newcommand\python[1]{{\pythonstyle\inlinestyle\lstinline!#1!}}


% TODOS
\iftoggle{greytext}{
\newcommand{\todo}[1]{\textbf{TODO:} #1}
}{
\newcommand{\todo}[1]{\textcolor{red}{\textbf{TODO:} #1}}
}

% CUSTOM COMMANDS
% TODO: fix the spacing issues
% package for conditional actions
\usepackage{xifthen}
% color definitions
\iftoggle{greytext}{
\definecolor{custtitlecolor}{rgb}{1, 1, 1}
\definecolor{defcolor}{rgb}{0.97,0.97,0.97}
\definecolor{thmcolor}{rgb}{0.97,0.97,0.97}
\definecolor{excolor}{rgb}{0.97,0.97,0.97}
\definecolor{importantcolor}{rgb}{0.97,0.97,0.97}
\definecolor{algcolor}{rgb}{0.97,0.97,0.97}
}{
\definecolor{custtitlecolor}{rgb}{0, 0, 0}
\definecolor{defcolor}{rgb}{0.75, 1, 0.75}
\definecolor{thmcolor}{rgb}{1, 0.75, 0.75}
\definecolor{excolor}{rgb}{1, 0.95, 0.43}
\definecolor{importantcolor}{rgb}{1, 0.55, 0}
\definecolor{algcolor}{rgb}{0.9, 0.9, 0.99}
}
% conditional checks
\newcommand{\emptyarg}[1][]{\ifthenelse{\isempty{#1}}{}{\ (#1)}}
% custom commands with optional argument
\newcommand{\Def}[1][]{%\colorbox{defcolor}
\kern+0.3ex}
\newcommand{\Thm}[1][]{%\colorbox{thmcolor}
\kern+0.3ex}
\newcommand{\Lem}[1][]{\colorbox{thmcolor}{%
\color{custtitlecolor}{\textbf{L.\emptyarg[#1]}}}\kern+0.3ex}
\newcommand{\Cor}[1][]{\colorbox{thmcolor}{%
\color{custtitlecolor}{\textbf{C.\emptyarg[#1]}}}\kern+0.3ex}
\newcommand{\Ex}[1][]{\colorbox{excolor}{%
\color{custtitlecolor}{\textbf{Ex.\emptyarg[#1]}}}\kern+0.3ex}
\newcommand{\Alg}[1][]{\colorbox{algcolor}{%
\color{custtitlecolor}{\textbf{Alg. #1}}}\kern+0.3ex}
% custom commands with no argument
\newcommand{\Com}{\textbf{Com.} }
\newcommand{\Note}{\textbf{Note.} }
\newcommand{\Important}{\textbf{Important.} }
\newcommand{\Attention}{\textbf{Attention.} }
\newcommand{\Proof}{\textbf{Proof.} }
\newcommand{\Intuition}{\textbf{Intuition.} }
\newcommand{\Trick}[1][]{\textbf{Trick.\emptyarg[#1]}\kern+0.3ex}


% CUSTOM TITLES
\usepackage[explicit]{titlesec}
\iftoggle{greytext}{
% grey out sections
\definecolor{sectioncolor}{rgb}{0.95,0.95,0.95}
\definecolor{sectionbarcolor}{rgb}{0.98,0.98,0.98}
\newcommand*{\mybox}[1]{%
    \noindent\colorbox{sectionbarcolor}{%
        \parbox{\dimexpr\columnwidth-2\fboxsep\relax}{%
            \textcolor{white}{#1}}}}
}{
% use blue color
\definecolor{sectioncolor}{rgb}{0,0,205}
\newcommand*{\mybox}[1]{%
    \noindent\colorbox{sectioncolor}{%
        \parbox{\dimexpr\columnwidth-2\fboxsep\relax}{%
            \textcolor{white}{#1}}}}
}
% Raised Rule Command:
% - arg 1 (optional) how high to raise the rule
% - arg 2 thickness of the rule
\newcommand{\raisedrulefill}[2][0ex]{\leaders\hbox{\rule[#1]{1pt}{#2}}\hfill}

% multi-line titles
\def\maxwidth{.78\linewidth}
\newsavebox\tempbox
\def\testwidth#1{%
\sbox{\tempbox}{#1}%
  \ifdim\wd\tempbox<\maxwidth
    \usebox{\tempbox}%
  \else
    \parbox{\maxwidth}{#1}%
  \fi}

% part title format
% TODO: scale-up the letter
\renewcommand\thepart{{\HUGE $\mathcal{\Alph{part}}$.}}
\titleformat{\part}{\Huge\bfseries}{\color{sectioncolor}\thepart\, #1}{0em}{}
% section title format
\makeatletter
\renewcommand\section{
\@startsection {section}{1}{\z@}{0pt}{0pt}{\normalfont\bfseries\mybox}}
\makeatother
% subsection title format
\titleformat{\subsection}{\bfseries}
    {\color{sectioncolor}\rule[0.3ex]{0pt}{0pt}
    \thesubsection\,\rule[0.3ex]{8pt}{1.5pt}\,}{0em}
    {\color{sectioncolor}\testwidth{#1}\,\raisedrulefill[0.3ex]{1.5pt}}
% subsubsection title format
\titleformat{\subsubsection}{\bfseries}
    {\color{sectioncolor}\rule[0.35ex]{0pt}{0pt}
    \thesubsubsection\,\rule[0.35ex]{8pt}{0.5pt}\,}{0em}
    {\color{sectioncolor}\testwidth{#1}\,\raisedrulefill[0.35ex]{0.5pt}}
% title spacings
\iftoggle{shortdocument}{
\titlespacing*{\part}
{0pt}{0ex plus 0ex minus 0ex}{0ex plus 0ex}
\titlespacing*{\section}
{0pt}{0ex plus 0ex minus 0ex}{0ex plus 0ex}
\titlespacing*{\subsection}
{0pt}{0ex plus 0ex minus 0ex}{0ex plus 0ex}
\titlespacing*{\subsubsection}
{0pt}{0ex plus 0ex minus 0ex}{0ex plus 0ex}
}{
\titlespacing*{\part}
{0pt}{1.2ex plus 1ex minus .2ex}{0ex plus .2ex}
\titlespacing*{\section}
{0pt}{0.8ex plus 1ex minus .2ex}{0ex plus .2ex}
\titlespacing*{\subsection}
{0pt}{0.7ex plus 1ex minus .2ex}{0ex plus .2ex}
\titlespacing*{\subsubsection}
{0pt}{0.5ex plus 1ex minus .2ex}{0ex plus .2ex}
}

% TABLE OF CONTENT
% remove table of contents title
\makeatletter
\renewcommand\tableofcontents{%
    \@starttoc{toc}%
}
\makeatother

% SEPARATORS
\usepackage{dashrule}
\newcommand{\sep}{\vspace{0pt}\noindent\hrule\vspace{0pt}}
\newcommand{\ssep}{\hdashrule[1.1ex]{\linewidth}{0.1pt}{0.3mm}}

% INTER-TITLES
\newcommand{\intertitle}[1]{\rule[0.35ex]{5pt}{0.5pt}
\textbf{#1} 
\raisedrulefill[0.35ex]{0.5pt}
}

% TITLE
\title{Machine Learning}
\author{Summary - Fall 2017}
\date{Andreas Bloch}

\begin{document}
\begin{landscape}


\pagenumbering{gobble}

% GREY TEXT FOR SUMMARIES
\iftoggle{greytext}{%
  % grey out text
  \definecolor{lighttext}{rgb}{0.95,0.95,0.95}
  \color{lighttext}
}{
  % show text with normal color
}

% EQUATION SPACING
\iftoggle{shortdocument}{
\setlength{\abovedisplayskip}{0pt}%
\setlength{\belowdisplayskip}{0pt}%
\setlength{\abovedisplayshortskip}{0pt}%
\setlength{\belowdisplayshortskip}{0pt}%
\setlength{\jot}{0pt}% Inter-equation spacing
}{
\setlength{\abovedisplayskip}{3pt}%
\setlength{\belowdisplayskip}{3pt}%
\setlength{\abovedisplayshortskip}{3pt}%
\setlength{\belowdisplayshortskip}{3pt}%
\setlength{\jot}{3pt}% Inter-equation spacing
}

\begin{multicols*}{3}
\raggedcolumns

\section{Probability}

\begin{comment}
Sum Rule \(P(X=x_i) = \sum_{j=1}^{J} p(X=x_i,Y=y_i)\)

Product rule \(P(X, Y) = P(Y|X) P(X)\)

Independence \(P(X, Y) = P(X)P(Y)\)

Bayes' Rule \(P(Y|X) = \frac{P(X|Y)P(Y)}{P(X)} = 
\frac{P(X|Y)P(Y)}{\sum\limits^k_{i=1}P(X|Y_i)P(Y_i)}\)
\end{comment}

Cond. Ind. 
$
\rX\perp\rY|Z
\Longrightarrow
\cProb{X,Y}{Z}=\cProb{X}{Z}\cProb{Y}{Z}
$

Cond. Ind.
$
\rX\perp\rY|Z
\Longrightarrow
\cProb{X}{Y,Z}=\cProb{X}{Z}
$

\sep

$\Exp{\rX}=\int_{\cX} t \cdot f_X(t)\,dt=:\mu_X$

\begin{gather*}
\begin{align*}
\textstyle{
\Var{X}
=\Exp{\left(X-\Exp{X}\right)^2}
=\int_{\cX} (t-\Exp{\rX})^2 f_X(t)\,dt
=
\Exp{X^2}-\Exp{X}^2
}
\end{align*}
\end{gather*}

$
\Cov{X,Y}=\int_{\cX}\int_{\cX}f(x,y)(x-\mu_x)(y-\mu_x)
\,dx\,dy
$

\sep

``$\MX^2=\MX\MX^T$''$\geq 0$ ((symmetric) positive semidefinite)

$
\Var{X}=\Exp{X^2}-\Exp{X}^2
$

$
\Var{\MA\MX}=\MA\Var{\MX}\MA^\T
$
\quad
$\Var{aX+b}=a^2\Var{X}$

\begin{gather*}
\begin{align*}
\textstyle{
\Var{\sum_{i=1}^n a_iX_i}
=
\sum_{i=1}^n a_i^2\Var{X_i}
+
2\sum_{\substack{i,j},i<j}a_ia_j\Cov{X_i,X_j}
}
\end{align*}
\end{gather*}

\begin{gather*}
\begin{align*}
\textstyle{
\Var{\sum_{i=1}^n a_iX_i}
=
\sum_{i=1}^n a_i^2\Var{X_i}
+
\sum_{\substack{i,j},i\neq j}a_ia_j\Cov{X_i,X_j}
}
\end{align*}
\end{gather*}

\sep

$
\frac{\partial}{\partial t} \Prob{X\leq t}
=
\frac{\partial}{\partial t} F_X(t)
=
f_X(t)
$
(derivative of c.d.f. is p.d.f)

$
f_{\alpha Y}(z)
=
\frac{1}{\alpha}f_Y(\frac{z}{\alpha})
$

\sep

Empirical CDF: $\hat{F}_n(t)=\frac{1}{n}\sum_{i=1}^n \ind{X_i\leq t}$

Empirical PDF: $\hat{f}_n(t)=\frac{1}{n}\sum_{i=1}^n\delta(t-X_i)$ (continuous)

Empirical PDF: $\hat{p}_n(t)=\frac{1}{n}\card{x=t}{x\in D}$ (discrete)

\sep

\Thm The MGF $\psi_{X}(t)=\Exp{e^{tX}}$ characterizes the distr. of a
rv

\begin{tabular}{@{}l@{}l|l@{}l@{}}
$Be(p)$: & $pe^t+(1-p)$
&
$\cN(\mu,\sigma)$: & $\exp\left(\mu t + \frac{1}{2}\sigma^2 t^2\right)$
\\
$Bin(n,p)$:\, & $(pe^t + (1-p))^n$\,
&
$Gam(\alpha,\beta)$:\, & $\left(\frac{1}{a-\beta t}^\alpha\right)$ for
$t<1/\beta$
\\
$Pois(\lambda)$: & $e^{\lambda(e^t -1)}$
\\
\end{tabular}

\sep

\Thm If $\rX_1,\ldots,\rX_n$ are ind. rvs with 
MGFs $M_{\rX_i}(t)=\Exp{e^{t\rX_i}}$, then the
MGF of $\rY=\sum_{i=1}^n a_i\rX_i$
is $M_{\rY}(t)=\prod_{i=1}^n M_{X_i}(a_it)$.

\sep

\Thm Let $X$, $Y$ be ind., then the p.d.f. of $Z=X+Y$ is the conv.
of the p.d.f. of $X$ and $Y$:

$
f_Z(z) = \int_{\R} f_X(t)f_Y(z-t) \,dt
= \int_{\R} f_X(z-t)f_Y(t) \,dt
$

\sep

$
\cN(\vx;\vmu,\MSigma)
=
\frac{1}{\sqrt{(2\pi)^d\det(\MSigma)}}
\exp\left(-\frac{1}{2}(\vx-\vmu)^\T\MSigma^{-1}
(\vx-\vmu)\right)
$

\begin{comment}
$
\cN(\vx;\vmu,\MSigma^{-1})
\propto
\exp\left(\frac{1}{2}(\vx-\vmu)^\T\MSigma(\vx-\vmu)\right)
$
\end{comment}

$
\hat{\mu}=\frac{1}{n}\sum_{i=1}^n \vx_i
$
\quad
$
\hat{\MSigma}=\frac{1}{n}\sum_{i=1}^n
(\vx-\hat{\vmu})(\vx-\hat{\vmu})^\T
$

\sep

\Thm
$
\Prob{
\begin{bmatrix}
\va_1\\
\va_2\\
\end{bmatrix}
}
=
\cDist{\cN}{
\begin{bmatrix}
\va_1\\
\va_2\\
\end{bmatrix}
}{
\begin{bmatrix}
\vu_1\\
\vu_2\\
\end{bmatrix},
\begin{bmatrix}
\MSigma_{11} & \MSigma_{12}\\
\MSigma_{21} & \MSigma_{22}
\end{bmatrix}
}
$\\
$\va_1,\vu_1\in\R^{e}$, $\MSigma_{11}\in\R^{e\times e}$ p.s.d.
$\MSigma_{12}\in\R^{e\times f}$ p.s.d.\\
$\va_2,\vu_2\in\R^{f}$, $\MSigma_{22}\in\R^{f\times f}$ p.s.d.
$\MSigma_{21}\in\R^{f\times e}$ p.s.d.\\
\begin{gather*}
\begin{align*}
\cProb{\va_2}{\va_1=\vz}
=
\cDist{\cN}{
\va_2
}{
\vu_2+\MSigma_{21}\MSigma_{11}^{-1}(\vz-\vu_1)
,
\MSigma_{22}-\MSigma_{21}\MSigma_{11}^{-1}\MSigma_{12}
}
\end{align*}
\end{gather*}

\sep

\Thm[Chebyshev] Let $X$ be a rv with $\Exp[X]=\mu$ and variance
$\Var{X}=\sigma^2<\infty$. Then for any $\epsilon > 0$, we have
$
\Prob{\abs{X-\mu}\geq\epsilon} \leq\frac{\sigma^2}{\epsilon^2}.
$

\sep

\Thm[Cramer Rao Bound] If $\hat{\vtheta}$ is unb., cons. est., then

$
MSE(\hat{\theta})=\Exp{\left(\hat{\theta}-\theta\right)^2}\geq
\frac{1}{\cI_n(\theta)}>
0, \quad\text{where}
$

$
\cI_n(\theta)=-\sum_{i=1}^n\Exp{\frac{\partial^2
\log\left(\cProb{x_i}{\vtheta}\right)}{\partial \theta ^2}}.
\quad\text{(Fisher Information)}
$

\sep

\Def[Conditional Expected Risk] Given rv $X$

$
R(f,X) = \int_{\cY} L(Y,f(X)) \cProb{Y}{X} \,dY
$

\sep

\Def[Total Expected Risk] for rvs $X,Y$
\begin{gather*}
\begin{align*}
\textstyle
R(f)&=\Exp[X]{R(f,X)}=\textstyle\int_{\cX} R(f,X) \Prob{X} \, dX
=\Exp[X,Y]{L(Y,f(X))}.
\end{align*}
\end{gather*}

\section{Matrix Calculus}

$
f(\vx)=f(\vx_0) + \nabla_{\vx}f(\vx_0)^\T(\vx-\vx_0)
$\quad (2nd order Taylor Expan.)

$\quad\qquad+\frac{1}{2}(\vx-\vx_0)^\T\text{Hess}(f;\vx_0)(\vx-\vx_0)+
\BigO((\vx-\vx_0)^3)
$

$
\text{Hess}(f;\vx_0)=\frac{\partial^2 f}{\partial \vx\partial\vx^\T}(\vx_0)
$

\sep

$
\frac{\partial}{\partial \vx}\left[\MA\vx\right] = \MA^\T
$
\quad
$
\frac{\partial}{\partial \vx}\left[\MA\vf(\vx)\right]
=\MA^\T\frac{\partial}{\partial \vx}\left[\vf(\vx)\right]
$

$
\frac{\partial}{\partial \vx}\left[\vx^\T\MA\vx\right] = 2\MA^\T\vx
$
\quad
$
\frac{\partial}{\partial \vx}\left[\norm{\vf(\vx)}_2^2\right]
=
2
\frac{\partial}{\partial \vx}\left[\vf(\vx)\right]
\vf(\vx)
$

\sep

\Thm[Sylvester Criterion] A $d\times d$ matrix is positive semi-definite if and
only if all the upper left $k\times k$ for $k=1,\ldots,d$ have a positive
determinant.

\section{Misc}

\Thm[Jensen] $f$ convex/\tcr{concave}, $\forall i\colon\lambda_i\geq 0$,
$\sum_{i=1}^n\lambda_i=1$

$
f\left(\sum_{i=1}^n\lambda_i\vx_i\right)
\leq\mcr{/\geq}
\sum_{i=1}^n\lambda_if\left(\vx_i\right)
$

Special case: $f(\Exp{X})\leq \Exp{f(X)}$.

\sep

\Thm[CSU] $\abs{\scprod{\vx,\vy}}^2\leq\scprod{\vx,\vx}\cdot\scprod{\vy,\vy}$

Special case: $(\sum x_i y_i)^2\leq(\sum x_i^2)(\sum y_i)^2$

Special case: $\Exp{XY}^2\leq\Exp{X^2}\Exp{Y^2}$

\sep

\textbf{Lag.:} $f(x,y)\, s.t.\, g(x,y) = c$
\,
$
\mathcal{L}(x, y, \gamma) = f(x,y) - \gamma ( g(x,y)-c)
$

\section{Kernels}

\Def[Kernel] Kernel functions $k(x,x')$ must satisfy (cf. properties of
covariance matrices)
\begin{enumerate}
  \item Symmetry: $k(x,x')=k(x',x)$
  \item Positive semi-definiteness (continuous case):\\
  $\forall f\in
  L_2\,\forall\Omega\subset\R\colon
  \int_{\Omega}k(x,x')f(x)f(x')\,dx\,dx'\geq 0$
  \item Positive semi-definiteness (discrete case):\\
  For any $n\in\N$, and any set $S=\set{\vx_1,\ldots,\vx_n}$, the kernel (Gram)
  matrix $\MK=(k(\vx_i,\vx_j))_{i,j=1}^n$ must be positive semidefinite
  ($\forall\vx\colon\vx^\T\MK\vx\geq 0$).
\end{enumerate}

\sep

\Thm[Comp. Rules] $k_1+k_2$ ; $k_1\cdot k_2$ ; $c\cdot k_1$ ($c>0$) ; $f(k_1)$
where $f$ is a poly. w. pos. coeff., or the exp. func..

\sep

%\begin{center}
\begin{tabular}{ll}
\textbf{Kernel} $k(\vx,\vy)=$ & \textbf{Feature Map} $\vphi(\vx)$=\\
$k_1(\vx,\vy)+k_2(\vx,\vy)$ & $(\vphi_1(\vx),\vphi_2(\vx))^\T$\\
$c\cdot k_1(\vx,\vy)$ ($c>0$) & $\sqrt{c}\cdot\vphi_1(\vx)$\\
$k_1(\vx,\vy)\cdot k_2(\vx,\vy)$ & 
$(\phi_{1}(\vx)_i\cdot\phi_2(\vy)_j)^\T_{\text{ for }1\leq i\leq d_1, 1\leq
j\leq d_2}$
\\
$\vx^\T\MA\vy$ $\MA$ p.s.d. & $\ML^\T\vx$, $\MA=\ML\ML^\T$\\
$\vx^\T\MM^\T\MM\vx$, $\MM$ arbitrary & $\MM\vx$
\end{tabular}
%\end{center}

\sep

Linear \(k(x,y) = x^\top y\), 
Polynomial \(k(x,y) = (x^\top y + 1)^d\)

RBF \(k(x,y) = \exp(\frac{-\|x-y\|^2_2}{h^2})\),
Sigmoid \(k(x,y) = \tanh(k x^\top y - b)\)

\sep

Proof Tricks: Counterexample, Prove by ``for arbitrary $n$, $x_1,\ldots,x_n$,
$c_1,\ldots,c_n$ $\sum_i\sum_jc_ic_kk(x_i,x_j)\geq 0$'', constant kernel
$k(x,y)=c$ is a kernel, find feature map and inner product.

\section{Regression}

\textbf{Problem} What is the optimal estimate of a function $f\colon\R^d\to\R$
based on noisy data
$
y_i=f(x_i)+\epsilon_i
$

\textbf{Solution} the regression function
\[
f^*(x)=\cExp{Y}{X=x}=\textstyle\int_{\cY}y\cdot\cProb{y}{X=x}\,dy
\]

\subsection{Ridge Regression}

$\vw^*=\argmin_{\vw}\norm{\vy-\MX\vw}_2^2 + \lambda \norm{\vw}_2^2 \quad
(\lambda > 0, \text{ chosen via CV}) $

$\vw^*=(\MX^\T\MX+\lambda\MI)^{-1}\MX^\T\vy$ \quad (always has a solution)

$\vg_t =-2\MX^\T(\vy-\MX\vw_t) +2\lambda\vw_t$

Note: Scale of the data matters for $\lambda$! ($\rightarrow$ normalize
data)

\sep

$Y\sim\cN(\vw^\T\vx,\sigma^2)$,\quad $y_i=\vw^\T\vx_i+\epsilon$, \quad
$\epsilon\sim\cN(0,\sigma^2)$

$\cProb{Y=y}{\vx=\vx,\vtheta=(\vw,\sigma^2)}=\cN(y;h(\vx),\sigma^2)$,  
$h(\vx)=\vw^\T\vx$

Weights prior:
$\vw\sim\cN(0,\beta^2\MI)$,\quad $w_i\sim\cN(0,\beta^2)$

Maximizing $\cProb{\vw}{D}$ then leads to the connection
$\lambda=\frac{\sigma^2}{\beta^2}$.

\subsection{Bayesian Linear Regression}

$
\rY=\rvX^\T\vbeta+\epsilon$, $\epsilon\sim\cN(0,\sigma^2)
$

$
\cProb{Y}{\rvX,\vbeta,\sigma}=\cN(\rvX^\T\vbeta,\sigma^2)
\propto\exp\left(-\frac{1}{2\sigma^2}(\rY-\rvX^\T\vbeta)^2\right)
$

$
\cProb{\vbeta}{\MLambda}=\cN(\vo,\MLambda^{-1})
\propto
\exp\left(-\frac{1}{2}\vbeta^\T\MLambda\vbeta\right)
$

\begin{gather*}
\begin{align*}
\cProb{\vbeta}{\MX,\vy,\MLambda,\sigma}
=
\frac{\cProb{\vy}{\MX,\vbeta,\MLambda,\sigma}\cProb{\vbeta}{\MX,\MLambda,\sigma}}{
\cProb{\vy}{\MX,\MLambda,\sigma}}
\propto
\cProb{\vy}{\MX,\vbeta,\sigma}\cProb{\vbeta}{\MLambda}
\end{align*}
\end{gather*}

$
=\prod_{i=1}^n\cProb{y_i}{\vx_i,\vbeta,\sigma}\cProb{\vbeta}{\MLambda}
=\cN(\vbeta;\vmu_{\vbeta},\MSigma_{\vbeta})
$

$
\vmu_{\vbeta}=\left(\MX^\T\MX-\sigma^2\MLambda\right)^{-1}\MX^\T\vy,\quad
\MSigma_{\vbeta}=\sigma^2\left(\MX^\T\MX+\sigma^2\MLambda\right)^{-1}
$

Special Case: Ridge Regression: $\MLambda=\lambda\MI$, $\sigma=1$

\subsection{Kernelized Ridge Regression}

\textbf{Insight} optimal $\vw^*$ lies in the span of the data.

$\vw^*=\MX_\phi^\T\vz^*$ \, 
\, ($\MK=\MX_\phi\MX_\phi^\T\in\R^{n\times n}$)

$\vz^*=\argmin_{\vz} \snorm{\MK\vz-\vy}_2^2 + \lambda\vz^\T\MK\vz $

\textbf{1) Closed form} $\vz^*=(\MX_\phi\MX_\phi^\T+\lambda\MI)^{-1}\vy
=(\MK-\lambda\MI)^{-1}\vy$

\textbf{2) Gradient descent} $\vg_t=2\MK^\T(\MK\vz-\vy) + 2 \lambda\MK\vz$

\textbf{Prediction} $f(\vx)={\vw}^\T\phi(\vx)=\ldots=\sum_{i=1}^n
z_ik(\vx_i,\vx)$

\textbf{Bayesian Interpretation} Same as ridge regression, except that the
hypothesis class for $\cH$ for $h$ (comp. of mean) may be different.

\subsection{Sparse Regression: LASSO}

Requires $\norm{\vw}_1\leq s$. Prior: $w_i\sim
p(w_i;0,b)=\frac{1}{2b}\exp\left(-\frac{\abs{w_i-\mu}}{b}\right)$ where $\mu=0$,
(connection: $\lambda=\frac{2\sigma^2}{b}$).

\subsection{Ensemble of Regressions}

\Thm The average of $B$ unbiased estimators $\hat{f}_i$ remains unbiased.

$
\Exp{\frac{1}{B}\sum_{i=1}^B \hat{f}_i(X)} - \theta = 0.
$

\sep

\Thm The variance of am average of $B$ unbiased estimators, $(*)$ which have
small covariances $\approx 0$, and similar variances $\approx\sigma^2$ my reduce the
variance to $\sigma^2/B$.
\begin{gather*}
\begin{align*}
\Var{\frac{1}{B}\sum_{i=1}^B \hat{f}_i(X)}
=
\frac{1}{B^2}\sum_{i=1}^B\Var{ \hat{f}_i(X)}
+
\frac{1}{B^2}\sum_{\substack{i,j\\i\neq j}}^B\Cov{\hat{f}_i(X),\hat{f}_i(X)}
\stackrel{(*)}{\approx}\frac{\sigma^2}{B}.
\end{align*}
\end{gather*}

\subsection{Bias Variance for Squared Loss}

\begin{gather*}
\begin{flalign*}
&\Exp[D]{
 \Exp[\MX,Y]{(Y-\hat{h}_D(\MX))^2}
}
\qquad\left(\text{Note: }h^*(X)=\cExp[Y]{Y}{X}\right)
\\
&
=
\underbrace{
\Exp[\MX]{
\left(
\Exp[D]{\hat{h}_D(\MX)}-
h^*(\MX)
\right)^2}
}_{\text{Bias}^2}
+
\underbrace{
\Exp[\MX]{
\Exp[D]{\left(\hat{h}_D(\MX)-\Exp[D']{\hat{h}_{D'}(\MX)}\right)^2}}
}_{\text{Variance}}
\\
&+\Exp[\MX,Y]{\left(Y-h^*(\MX)\right)^2}
\qquad
\left(\text{Noise}\right)
\end{flalign*}
\end{gather*}

Derivation: 1) $\pm\cExp[Y]{Y}{X}$ gives noise, 2) Prove vanishing cross-term,
3) $\pm\Exp[D]{\hat{f}(X)}$ gives bias and variance

\subsection{Gaussian Processes}

$\rvY=\MX\vbeta+\vepsilon$, $\vepsilon\sim\cN_n(\vo,\sigma^2\MI_n)$,
\,
$\vbeta\sim\cN(\vo,\MLambda^{-1})$

Since the outputs $\vy$ are a linear combination of normally distributed rvs
$\vbeta$, they are jointly Gaussian themselves.

$
\Exp[\vbeta,\vepsilon]{\vy}
=
\Exp[\vbeta,\vepsilon]{\MX\vbeta+\vepsilon}
=
\MX\Exp[\vbeta]{\vbeta}
+\Exp[\vepsilon]{\vepsilon}
=
\MX\vo
+
\vo
=
\vo.
$

$\Cov{\vy}
=\Exp{\vy\vy^\T}
=\Exp{(\MX\vbeta+\vepsilon)(\MX\vbeta+\vepsilon)^\T}
=\MX\MLambda^{-1}\MX^\T+\sigma^2\MI.$

Special case: $\Lambda^{-1}:=\lambda\MI$:
$\Cov{\vy}=\lambda^{-1}(\MX\MX^\T+\lambda\sigma^2\MI_n)$.

Then we can rewrite the joint distr. as $\vy\sim\cN(\vo,\MK+\sigma^2\MI)$

\begin{comment}
$
\begin{pmatrix}
y_1\\
y_2\\
\vdots\\
y_n
\end{pmatrix}
\sim
\cN
\left(
\begin{pmatrix}
0\\
0\\
\vdots\\
0
\end{pmatrix},
\begin{pmatrix}
k_{1,1} + \sigma^2 & k_{1,2} & \cdots & k_{1,n}\\
k_{2,1} & k_{2,2} + \sigma^2 & \cdots & k_{2,n}\\
\vdots & \vdots & \ddots & \vdots\\
k_{n,1} & k_{n,2} & \cdots & k_{n,n}+ \sigma^2\\
\end{pmatrix}
\right),
$
\end{comment}

where $k(\vx_i,\vx_j)=\vx_i^\T\MLambda^{-1}\vx_j$ is a so-called kernel
function (could be replaced with others).

$
\text{Joint D.}\quad
\cProb{
\begin{pmatrix}
\vy\\
y_{n+1}
\end{pmatrix}}{
x_{n+1},\MX,\sigma
}
=
\cDist{\cN}{
\begin{pmatrix}
\vy\\
y_{n+1}
\end{pmatrix}
}{
\vo,
\begin{pmatrix}
\MC_n & \vk\\
\vk^\T & c
\end{pmatrix}
}
$\\
$\MK=k(\MX,\MX)$, 
$\MC_n=\MK+\sigma^2\MI$,
$c=k(x_{n+1},x_{n+1})+\sigma^2$,
$\vk=k(x_{n+1},\MX)$\\
$
\text{Pred D.}\quad
\cProb{y_{n+1}}{x_{n+1},\MX,\vy,\sigma}
=
\cDist{\cN}{
y_{n+1}
}{
\vmu_{y_{n+1}},
\sigma^2_{y_{n+1}}
}
$\\
$\vmu_{y_{n+1}}=\vk^\T\MC_n^{-1}\vy=\vk^\T(\MK+\sigma^2\MI)^{-1}\vy$,
$\sigma^2_{y_{n+1}}=c-\vk^\T\MC_{n}^{-1}\vk$

\section{Classification}

0/1 Loss  $w^* = \argmin_w \sum_{i=1}^n [y_i \neq sign(w^\top x_i)]$

Perceptron $w^* = \argmin_w \sum_{i=1}^n [\max(0, y_i w^\top x_i)]$

Exponential Loss $L(y,z)=\exp{-(2y-1)(2z-1)}$

Logistic Loss $L(y,z)=\ln(1+\exp{-(2y-1)(2z-1)})$

Dep. on appl. use $\vy\vw^\T\vx$ or $-(2y-1)(2z-1)$ as error

\subsection{Generative VS Discriminative}

\[
\cProb{Y}{X}=\frac{\cProb{X}{Y}\Prob{Y}}{\sum_y \cProb{X}{Y}\Prob{Y}}
\]
\textbf{Generative:} Est. both $\Prob{Y}$ and $\cProb{X}{Y}$ then use Bayes.

\textbf{Discr.:} Est. $\cProb{Y}{X}$ directly, fitting
discr. f. $g(Y,X)$.

\subsection{SVMs}

\subsubsection{Primal, constrained}

$
\min_{w} w^\top w + C \sum_{i=1}^{n} \xi_i, \text{ s.t. } y_i w^\top x_i \geq 1 - \xi_i, \xi_i \geq 0
$

\subsubsection{Primal, unconstrained}

$
\min_{w} w^\top w + C \sum_{i=1}^{n} \max(0, 1-y_i w^\top x_i) \text{ (hinge loss)}
$

\subsubsection{Dual}

$
\max_{\alpha} \sum_{i=1}^{n} \alpha_i - \frac{1}{2} \sum_{i,j} \alpha_i \alpha_j y_i y_j x_i^\top x_j, \text{ s.t. } 0 \leq \alpha_i \leq C
$

\subsubsection{Dual to Primal}

Dual to primal: $w^* = \sum_{i=1}^{n} \alpha^*_i y_i x_i$, $\alpha_i > 0$: support vector.

\subsection{Kernelized SVMs}

$
\max_{\alpha} \sum_{i=1}^{n} \alpha_i - \frac{1}{2} \sum_{i,j} \alpha_i \alpha_j y_i y_j k(x_i, x_j), \text{ s.t. } 0 \geq \alpha_i \geq C
$

Classify: $y = sign(\sum_{i=1}^{n} \alpha_i y_i k(x_i, x))$

\intertitle{How to find $a^T$?}

$a = \{w_0,w\}$ used along $\widetilde{x} = \{1,x\}$

Gradient Descent: $a(k+1) = a(k) - \eta(k) \nabla J(a(k))$

Newton: 2nd ord. Taylor, where $\eta_{opt} = H^{-1}$,\,
$H=\frac{\partial^2 J}{\partial a_i \partial a_j}$

$J$ is the cost mat., popular: 
Perceptron cost: $J_p (a) = \sum(-a^T \widetilde{x})$

\subsection{Logistic Regression}

$\cProb{Y=y}{\vx,\vw}=Ber(y;\sigma(\vw^\T\vx))
=\frac{1}{1+e^{-y\vw^\T\vx}}=p_y$

$\cProb{Y=+1}{\vx,\vw}=\sigma(\vw^\T\vx)=p_+$

Gr. st. w. Gau. Pr.:
$\vw\leftarrow\vw(1-2\lambda\eta_t) + \eta_t
y\vx \chProb{Y=-y}{\vw,\vx}$

\subsubsection{Multi-Class Logistic Regression}

$\cProb{Y=i}{\vx,\vw_1,\ldots,\vw_c}=\frac{\exp(\vw_i^\T\vx)}{
\sum_{j=1}^c \exp(\vw^\T_j\vx)
}=p_i
$

\section{Multiclass Classification}

\textbf{One-VS-All} $y=\argmax_{i\in\set{1,\ldots,c}}f_i(\vx)$

\textbf{One-VS-One} $\binom{c}{2}$ bin. clf. for each
$(i,j)\in\set{1,\ldots,c}^2$.

$f_{(i,j)}\colon\cX\to\set{-1,+1}$ \, 
$y=\argmax_{i\in\set{1,\ldots,c}}\sum_{\substack{j=1}{j\neq i}}^{c} 
\ind{f_{(i,j)}(\vx)=+1}$.

\begin{comment}
\section{Bootstrapping}

Sample creation with replacement. Compute mean error of samples, and variance.

Bootstrapping works if for $n \rightarrow \infty$ the error of
empirical\&bootstrap is the same as real\&empiricial

P. of sample not to appear.: $(1-\frac{1}{n})^n$. Goes to
$\frac{1}{e}$ for $n \rightarrow \infty$

Multiplicity N sample to choose k times with replacement: $\binom{N-1+k}{k}$.
In bootstrapping $N=k$
\end{comment}

\section{Jackknife}

Method to compensate for syst. est. errors (bias reduction).

\textbf{Goal:} Numerically estimate the bias of an estimator $\hat{S}_n$.

$
\tilde{S}_n=\frac{1}{n}\sum_{i=1}^n S^{-i}_{n-1}\quad \text{(LOO- Estimator)}
$

Then the est. for the bias of $\hat{S}_n$ is:
$
\text{bias}^{JK}=(n-1)(\tilde{S}_n-\hat{S}_n)
$

Then the unbiased estimate is $\hat{S}^{JK}=\hat{S}_n-\text{bias}^{JK}$.

\section{Probabilistic Methods}

$
\underbrace{\cProb{\text{model } \theta}{\text{data } D}}_{\text{Posterior}}
=
\frac{
\overbrace{\cProb{\text{data}}{\text{model}}}^{\text{Likelihood}}
\times
\overbrace{\Prob{\text{model}}}^{\text{Prior}}
}{
\underbrace{\Prob{\text{data}}}_{\text{Evidence}}
}
$

\subsubsection{Maximum (Cond.) LH Est., (MLE)}

\begin{gather*}
\begin{flalign*}
\vtheta^*&=\textstyle\argmax_{\vtheta}
\chProb{y_1,\ldots,y_n}{\vx_1,\ldots,\vx_n,\vtheta}\\
%&\stackrel{\text{i.i.d}}{=} \textstyle\argmax_{\vtheta} 
%\textstyle\prod_{i=1}^n
%\chProb{y_i}{\vx_i,\vtheta}
%= \textstyle\argmax_{\vtheta} \textstyle\sum_{i=1}^n
%\log\chProb{y_i}{\vx_i,\vtheta}\\
&= \textstyle\argmin_{\vtheta} -\textstyle\sum_{i=1}^n
\log\chProb{y_i}{\vx_i,\vtheta} = \textstyle\argmin_{\vtheta} \ldots 
\text{insert, derivate}.&
\end{flalign*}
\end{gather*}

\subsubsection{Maximum a Posteriori Estimate, (MAP)}
\begin{gather*}
\begin{flalign*}
\vtheta^*&=\textstyle\argmax_{\vtheta} \cProb{\vtheta}{D}
=\textstyle\argmax_{\vtheta}
\cProb{\vtheta}{\vx_1,\ldots,\vx_n,y_1,\ldots,y_n}\\
%&=\textstyle\argmax_{\vtheta} \frac{
%\Prob{\vtheta}
%\cProb{y_1,\ldots,y_n}{\vx_1,\ldots,\vx_n,\vtheta}
%}{
%\cProb{y_1,\ldots,y_n}{\vx_1,\ldots,\vx_n}
%}\\
%&=
%\textstyle\argmax_{\vtheta} \Prob{\vtheta}
%\cProb{y_1,\ldots,y_n}{\vx_1,\ldots,\vx_n,\vtheta}\\
%&=
%\textstyle\argmin_{\vtheta} -\log\Prob{\vtheta}
%-\log \cProb{y_1,\ldots,y_n}{\vx_1,\ldots,\vx_n,\vtheta}\\
&\stackrel{\text{\tiny i.i.d}}{=}
\textstyle\argmin_{\vtheta} -\log\Prob{\vtheta}
-\textstyle\sum_{i=1}^n\log \cProb{y_i}{\vx_i,\vtheta} 
= \text{\ldots insert, derivate}&
\end{flalign*}
\end{gather*}

\section{Ensemble Methods}

Use combination of simple hyp. (weak lerners) that are sufficiently
diverse to prod. a valid sol. with low bias and var.

\textbf{Bagging}: train weak l. on bootstr. sets with equal weights. \\

\textbf{Boosting}: train on all data, but reweigh misclassified samples higher
and use error-sensitive reweighting of classifiers.

\begin{comment}
\begin{algorithm}[H]
\caption{Bagging}
\DontPrintSemicolon
\KwIn: $\cZ=\set{(x_1,y_1),\ldots,(x_n,y_n)}$\;
\For{$b\gets 1$ \KwTo $B$}{
	$\cZ^{\star b}\gets b$-th bootstrap sample from $\cZ$ (i.i.d., with repl.)\;
	Construct classifer $c_b$ based on $\cZ^{\star b}$\;
}
\Return ensemble clf. $\hat{c}_B(x)=\sign(\sum_{i=1}^Bc_i(x))$ (majority)\;
\end{algorithm}
\end{comment}

\sep

Left out: Decision trees (Stump: $h(x) = sign(a x_i - t)$), Random forest:
bagging of trees.


\begin{comment}
\subsubsection{Decision Trees}
\textbf{Stumps}: partition linearly along 1 axis
$
h(x) = sign(a x_i - t)
$

\textbf{Decision Tree}: recursive tree of stumps, leaves have labels. To train,
either label if leaf's data is pure enough, or split data based on score.

\subsubsection{Random Forests}

Strategy for Bagging Trees: Loop: 1) Draw bootstrap sample, 2) Pick random
features $(\approx\sqrt{d})$, 3) pick best variable to split classes 4)
Recursively split until the minimal node size is reached. 5) Return ensemble of
decision trees.
\end{comment}
\subsubsection{Ada Boost}

$
f^*(x) = \argmin_{f\in F} \sum_{i=1}^{n} \exp(-y_i f(x_i))
\,\,
\text{(eff. mins. exp. l.)}
$

\begin{algorithm}[H]
\caption{AdaBoost Algorithm}
\DontPrintSemicolon
Initialize the observation weights: $\forall
i\colon w_i^{(1)}\gets\frac{1}{n}$\;
\For{$b\gets 1$ \KwTo $B$}{
	Fit a clf. $c_b(x)$ to the weigthed training data (acc. to $w_i$)\;
	$\epsilon_b\gets\frac{\sum_{i=1}^n w_i^{(b)}\ind{c_b(x_i)\neq y_i}}{\sum_{i=1}^n
w_i^{(b)}}$ \,\, (error of the weak learner)\;
	$\alpha_b\gets\log\left(\frac{1-\epsilon_b}{\epsilon_b}\right)$\,\,(weigh
	classifier acc. to accuracy)\; 
	Update the datapoint weights for the next step:\;
	For all $i\in\set{1,\ldots n}$ do
	$w_i^{(b+1)}\gets w_i^{(b)}\exp\left(\alpha_b\ind{y_i\neq c_b(x_i)}\right)$\;
}
\Return $\hat{c}_B(x)=\sign\left(\sum_{b=1}^B \alpha_bc_b(x)\right)$\;
\end{algorithm}

Additive log. reg., Bayesian approach (assumes
poster.), Newtonlike updates (GD), if prev. clf. bad,
next has high weight.

\section{Generative Methods}

\subsection{Naive Bayes}
Features indep.
$
P(y|x) = \frac{1}{Z} P(y) P(x|y), Z = \sum_{y} P(y) P(x|y)
$

$
y = \argmax_{y'} P(y'|x) = \argmax_{y'} \hat{P}(y') \prod_{i=1}^{d} \hat{P}(x_i|y')
$

Discr. Func.:
$
f(x) = \log(\frac{P(y=1|x)}{P(y==1|x)}), y=sign(f(x))
$

\subsection{Fischer's LDA}
$J(\vw)=\frac{\norm{\vw^\T(\vmu_1-\vmu_2)}}{\vw^\T(\MSigma_1 + \MSigma_2)\vw}$
$\hat{\vw}=(\MSigma_1+\MSigma_2)^{-1}(\vmu_1-\vmu_2)$ (unscaled)

However, all samples influence boundary (better: points at border, SVM)

\section{Unsupervised Learning}

\subsection{Gaussian Mixture Modeling}
$
(\mu^*, \Sigma^*, \pi^*) = \argmin -\sum_i log \sum_{j=1}^{k} \pi_j
\mathcal{N}(x_i|\mu_i,\Sigma_j) $

\subsection{EM Algorithm}
Problem: sum within log-term of likelihood.
 
\textbf{E-step}: expectation: pick clusters for points.

Calculate $\gamma_j^{(t)}(x_i) = \frac{P(c|\theta^j) (x_i|c,\theta^j)}{\sum P(x_i|\theta)}$ for each $i$ and $j$

\textbf{M-Step}: maximum LH: adjust clusters to best fit points.



$
\text{prior}_j^{(t)} = \pi^{(t)}_j \leftarrow \frac{1}{n}\sum_{i=1}^n
\gamma_j^{(t)}(x_i) 
$

$
\mu_j^{(t)} \leftarrow \frac{\sum_{i=1}^n \gamma_j^{(t)}(x_i)(x_i)}{\sum_{i=1}^n \gamma^{(t)}(x_i)} 
$,
$
\Sigma^{(t)}_j \leftarrow \frac{\sum_{i=1}^n
\gamma_j^{(t)}(x_i)(x_i-\mu_j^{(t)})(x_i-\mu_j^{(t)})^\top}{\sum_{i=1}^n \gamma_j^{(t)}(x_i)} $

\begin{comment}
\subsection{EM for non-gaussian mixture}
\textbf{Problem setting}
Derive the EM algorithm. The model behind the EM in the form:
\(p(x) = \sum\limits^n_{j=1}\pi_j Pr_{\text{non-gaussian}}\)
\textbf{Solution}
\begin{enumerate}[topsep=0pt,itemsep=0ex,partopsep=1ex,parsep=1ex,leftmargin=*]
\item Write  log likelihood function:
\[L(X,\{\text{params}Pr_{non-gaussian} u_js\}) = \sum\limits^n_{i=1} \log \left( p(x_i)\right)\]
\item Write \(\gamma_j(x_i) = p(z_j = j | x_i)\) (prob. that \(x_i\) was generated from \(j^{th}\) dist. of the mixture).
\begin{align*}
\gamma_j(x_i) &= p(z_i = j | x_i) = \frac{p(x_i|z_i = j) p(z_i = j)}{p(x_i)}\\
&= \frac{p(x_i|z_i = j) p(z_i = j)}{\sum\limits^n_{k=1}p(x_i|z_i = k) p(z_i = k)}
= \frac{\pi_i L(X,u_js)}{\sum^n_{k=1}\pi_k L(X,u_js)}
\end{align*}
\item \textbf{To get optimal }\(\mathbf{u_js}\), derive \(L(x,u_js)\) by each param \(u_j\).
You get:
\[\nabla x_j L(X,u_js) = \sum\limits^n_{i=1}\frac{1}{Pr_{non-gaussian}} \cdot
(\nabla u_j Pr_{non-gaussian})\](possibly replace by \(\gamma_j(x_i)\) (above
and below) leaving back some factor). Solve for the parameter \(u_j\).
\item \textbf{Estimate the \(\pi\) parameters} by Lagrange optimization on log
likelihood function and constraint \(\lambda \left(\sum\limits^k_{j=1}\pi_j
-1\right) \). Put the \(\lambda\) into the formula and find the formula for
\(\pi_j\).
\end{enumerate}
\end{comment}

\end{multicols*}
\end{landscape}
\end{document}
